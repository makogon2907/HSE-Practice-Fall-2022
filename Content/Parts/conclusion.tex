%!TEX root = ../main

\section{Заключение}

Исходя из содержания, становится ясно, что распределенные системы обучения нейронных сетей ползволяют получить ранее недостигаемые результаты. Так как нейронные сети могут сильно различаться по количеству параметров, по типу слоёв и тому подобному, каждая представленная система лучше справляется именно с теми задачами, для которых она разработана. При этом, из общего, всем таким системам свойственно добавление накладных расходов и слабая консистентность данных, но в конечном счете увеличение самого главного — скорости достижения моделью целевой точности, что может быть не просто удобным механизмом для быстрого обучения, но и необходимостью, в случаях когда модель не может обучиться на одной машине за разумное время. Таким образом, использование распределённых систем становится неотъемлемым компонентом в машинном обучении, позволяющим справляться с ранее нерешаемыми задачами.
